%%%%%%%%%%%%%%%%%%%%%%%%%%%%%%%%%%%%%%%%%
% Journal Article
% LaTeX Template
% Version 1.4 (15/5/16)
%
% This template has been downloaded from:
% http://www.LaTeXTemplates.com
%
% Original author:
% Frits Wenneker (http://www.howtotex.com) with extensive modifications by
% Vel (vel@LaTeXTemplates.com)
%
% License:
% CC BY-NC-SA 3.0 (http://creativecommons.org/licenses/by-nc-sa/3.0/)
%
%%%%%%%%%%%%%%%%%%%%%%%%%%%%%%%%%%%%%%%%%

%----------------------------------------------------------------------------------------
%	PACKAGES AND OTHER DOCUMENT CONFIGURATIONS
%----------------------------------------------------------------------------------------

\documentclass[twoside,twocolumn]{article}

%\usepackage{booktabs,caption}
\usepackage[flushleft]{threeparttable}
\usepackage{graphicx}
\usepackage{blindtext} % Package to generate dummy text throughout this template 

\usepackage[sc]{mathpazo} % Use the Palatino font
\usepackage[T1]{fontenc} % Use 8-bit encoding that has 256 glyphs
\linespread{1.05} % Line spacing - Palatino needs more space between lines
\usepackage{microtype} % Slightly tweak font spacing for aesthetics

\usepackage[english]{babel} % Language hyphenation and typographical rules

\usepackage[hmarginratio=1:1,top=32mm,columnsep=20pt, left= 2.35cm, right = 2.35cm, footskip = 2cm]{geometry} % Document margins
\usepackage[hang, small,labelfont=bf,up,textfont=it,up]{caption} % Custom captions under/above floats in tables or figures
\usepackage{booktabs} % Horizontal rules in tables

\usepackage{lettrine} % The lettrine is the first enlarged letter at the beginning of the text

\usepackage{enumitem} % Customized lists
\setlist[itemize]{noitemsep} % Make itemize lists more compact

\usepackage{abstract} % Allows abstract customization
\renewcommand{\abstractnamefont}{\normalfont\bfseries} % Set the "Abstract" text to bold
\renewcommand{\abstracttextfont}{\normalfont\small\itshape} % Set the abstract itself to small italic text

\usepackage{titlesec} % Allows customization of titles
\renewcommand\thesection{\Roman{section}} % Roman numerals for the sections
\renewcommand\thesubsection{\roman{subsection}} % roman numerals for subsections
\titleformat{\section}[block]{\large\scshape\centering}{\thesection.}{1em}{} % Change the look of the section titles
\titleformat{\subsection}[block]{\large}{\thesubsection.}{1em}{} % Change the look of the section titles

\usepackage{fancyhdr} % Headers and footers
\pagestyle{fancy} % All pages have headers and footers
\fancyhead{} % Blank out the default header
\fancyfoot{} % Blank out the default footer
%\fancyhead[C]{Running title $\bullet$ May 2016 $\bullet$ Vol. XXI, No. 1} % Custom header text
\fancyfoot[RO,RE]{\thepage} % Custom footer text

\usepackage{titling} % Customizing the title section

\usepackage{hyperref} % For hyperlinks in the PDF

%----------------------------------------------------------------------------------------
%	TITLE SECTION
%----------------------------------------------------------------------------------------

\setlength{\droptitle}{-4\baselineskip} % Move the title up

\pretitle{\begin{center}\huge\bfseries} % Article title formatting
\posttitle{\end{center}} % Article title closing formatting
\title{Studies of Phase Transitions in Magnetic Fields } % Article title

\author{%
\textsc{Andreas Fagerheim}\thanks{\url{https://github.com/AndreasFagerheim/Project-4}} \\[1ex] % Your name
\normalsize Department of Physics, University of Oslo, Norway \\ % Your institution
%\normalsize \href{mailto:john@smith.com}{john@smith.com} % Your email address
%\and % Uncomment if 2 authors are required, duplicate these 4 lines if more
%\textsc{Jane Smith}\thanks{Corresponding author} \\[1ex] % Second author's name
%\normalsize University of Utah \\ % Second author's institution
%\normalsize \href{mailto:jane@smith.com}{jane@smith.com} % Second author's email address
}
\date{\today} % Leave empty to omit a date

%---------------------------------------------------------------------------------------
\renewcommand{\maketitlehookd}{%
\begin{abstract}

This article set forth to integrate a six-dimensional integral which is 
used to determine the ground state
correlation energy between two electrons in a helium atom.  The
integral appears in many quantum mechanical applications. We will first solve the integral through a brute force manner using Gauss Quadrature (Gauss-Legendre) and Monte Carlo method. By changing coordinate frame and choosing a better suited probability distribution function (PDF) we implemented improved methods respectively Gauss-Laguerre and Monte Carlo with Importance Sampling. This article will show that by doing these implementations of improvement we can get better precision off our numerical methods for integration. All programs developed and used are available in the github link in the footnote.


\end{abstract}
}

%----------------------------------------------------------------------------------------

\begin{document}
\maketitle

\section{Introduction}

The aim of this project is to study a widely popular model to simulate phase transitions, the so-called Ising model in two dimensions. At a given critical temperature, this model exhbits a phase transition from a magnetic phase (a system with a finite magnetic moment) to a phase with zero magnetization.
This is a so-called binary system where the objects at each lattice site can only take two values. These could be $0$ and $1$ or other values. Here we will use spins pointing up or down as the model for our system. But we could replace the spins with blue and green balls for example.
The \href{{https://en.wikipedia.org/wiki/Ising_model}}{Ising model} has been extremely popular, with applications spanning from studies of phase transitions to simulations in statistics. In one and two dimensions its has analytical solutions to several expectation values and it gives a qualitatively  good underatanding of several types of phase transitions.  



\section{Theory}

In its simplest form
the energy of the Ising model is expressed as, without an externally applied magnetic field, 
\[
E=-J\sum_{< kl >}^{N}s_ks_l 
\]
with
$s_k=\pm 1$. The quantity $N$ represents the total number of spins and $J$ is a coupling
constant expressing the strength of the interaction between
neighboring spins.  The symbol $<kl>$ indicates that we sum over
nearest neighbors only. We will assume that we have a ferromagnetic
ordering, viz $J> 0$.  We will use periodic boundary conditions and
the Metropolis algorithm only. The material on the Ising model can be found in chapter 13 of the lecture notes. The Metropolis algorithm is discussed in chapter 12.

\subsection{Studies of phase transitions.}
Near $T_C$ we can characterize the behavior of many physical quantities
by a power law behavior.
As an example, for the Ising class of models, 
the mean magnetization is given by
\[
  \langle M(T) \rangle \sim \left(T-T_C\right)^{\beta},
\]
where $\beta=1/8$ is a so-called critical exponent. A similar relation
applies to the heat capacity

\[
  C_V(T) \sim \left|T_C-T\right|^{\alpha},
\]
and the susceptibility
\begin{equation}
  \chi(T) \sim \left|T_C-T\right|^{\gamma},
\end{equation}
with $\alpha = 0$ and $\gamma = 7/4$.
Another important quantity is the correlation length, which is expected
to be of the order of the lattice spacing for $T>> T_C$. Because the spins
become more and more correlated as $T$ approaches $T_C$, the correlation
length increases as we get closer to the critical temperature. The divergent
behavior of $\xi$ near $T_C$ 
is

\begin{equation}
  \xi(T) \sim \left|T_C-T\right|^{-\nu}.
  \label{eq:xi}
\end{equation}
A second-order phase transition is characterized by a
correlation length which spans the whole system.
Since we are always limited to a finite lattice, $\xi$ will
be proportional with the size of the lattice. 
Through so-called finite size scaling relations
it is possible to relate the behavior at finite lattices with the 
results for an infinitely large lattice.
The critical temperature scales then as

\begin{equation}
 T_C(L)-T_C(L=\infty) = aL^{-1/\nu},
 \label{eq:tc}
\end{equation}
with  $a$ a constant and  $\nu$ defined in Eq. (\ref{eq:xi}).
We set $T=T_C$ and obtain a mean magnetisation

\begin{equation}
  \langle {\cal M}(T) \rangle \sim \left(T-T_C\right)^{\beta}
  \rightarrow L^{-\beta/\nu},
  \label{eq:scale1}
\end{equation}
a heat capacity

\begin{equation}
  C_V(T) \sim \left|T_C-T\right|^{-\gamma} \rightarrow L^{\alpha/\nu},
  \label{eq:scale2}
\end{equation}
and susceptibility

\begin{equation}
  \chi(T) \sim \left|T_C-T\right|^{-\alpha} \rightarrow L^{\gamma/\nu}.
  \label{eq:scale3}
\end{equation}
\section{Methods}


%------------------------------------------------

\section{Results}



\section{Conclusion}

%----------------------------------------------------------------------------------------
%	REFERENCE LIST
%----------------------------------------------------------------------------------------

\begin{thebibliography}{99} % Bibliography - this is intentionally simple in this template
%A statement requiring citation \cite{Hjorth-Jensen:2015dg}.
\bibitem[Hjorth-Jensen, 2015]{Hjorth-Jensen:2015dg}
Hjort-Jensen, M. (2015).
\newblock Computational Physics.
\bibitem[Hjorth-Jensen]{Hjorth-Jensen}
Hjort-Jensen, M.
\newblock https://github.com/CompPhysics/ComputationalPhysics

 
\end{thebibliography}

%----------------------------------------------------------------------------------------

\end{document}
